%!TEX TS-program = xelatex
%!TEX encoding = UTF-8 Unicode
% Awesome CV LaTeX Template for Cover Letter
%
% This template has been inspired by:
% https://github.com/posquit0/Awesome-CV
%
% Template license:
% CC BY-SA 4.0 (https://creativecommons.org/licenses/by-sa/4.0/)
%
% This class has been downloaded from:
% https://github.com/marckit/cv-marckit
%
% Author:
% Marc-André Labelle <marckitmail@gmail.com>


%-------------------------------------------------------------------------------
% CONFIGURATIONS
%-------------------------------------------------------------------------------
% A4 paper size by default, use 'letterpaper' for US letter
\documentclass[french, letterpaper]{marckit-cv}
\usepackage[ddmmyyyy]{datetime} % -> fr

% Configure page margins with geometry
\geometry{left=1.4cm, top=.8cm, right=1.4cm, bottom=1.8cm, footskip=.5cm}

% Specify the location of the included fonts
\fontdir[fonts/]

% Color for highlights
% My Favorite Colors: awesome-blue
\colorlet{awesome}{awesome-orange}
% Uncomment if you would like to specify your own color
% \definecolor{awesome}{HTML}{CA63A8}

% Colors for text
% Uncomment if you would like to specify your own color
% \definecolor{darktext}{HTML}{414141}
% \definecolor{text}{HTML}{333333}
% \definecolor{graytext}{HTML}{5D5D5D}
% \definecolor{lighttext}{HTML}{999999}

% Set false if you don't want to highlight section with awesome color
\setbool{acvSectionColorHighlight}{true}

% If you would like to change the social information separator from a pipe (|) to something else
\renewcommand{\acvHeaderSocialSep}{\quad\textbar\quad}


%-------------------------------------------------------------------------------
%	PERSONAL INFORMATION
%	Comment any of the lines below if they are not required
%-------------------------------------------------------------------------------
% Available options: circle|rectangle,edge/noedge,left/right
\photo[circle,noedge,left]{./images/profile}
\name{Marc-André}{Labelle}
%\position{Food Process Junior Engineer{\enskip\cdotp\enskip}Computer Science Student} % -> en
\position{Ingénieur junior en génie alimentaire{\enskip\cdotp\enskip}Étudiant en informatique et génie logiciel} % -> fr
\address{5784, 7e avenue, Montreal, QC H1Y 2N8 Canada}
\mobile{+1 (514) 568-6836}
\email{marckitmail@gmail.com}
\github{marckit}
\gitlab{marckit}
\linkedin{marckit}

%\quote{"The most important intelligence is not artificial"} % -> en
\quote{"L'intelligence la plus importante n'est pas artificiel"} % -> fr


%-------------------------------------------------------------------------------
%	LETTER INFORMATION
%	All of the below lines must be filled out
%-------------------------------------------------------------------------------
% The company being applied to
\recipient
%  {HR in charge}                       % -> en
%   {Compagny\\Street Address\\City}    % -> en
  {Au superviseur du brassage et à l'équipe des ressources humaines}   % -> fr
   {Sleeman Unibroue inc.\\80 Rue des Carrières\\ Chambly, QC J3L 2H6} % -> fr
% Date de compilation
\letterdate{25 mai 2020}
% The title of the letter
%\lettertitle{Information Technology} % -> en
\lettertitle{Application pour le poste de brasseur de bière}           % -> fr
% Ouverture de la lettre
%\letteropening{Dear Talent Acquisition Advisors,} % -> en
\letteropening{Monsieur ou Madame resposable de l'acquisition de talents,} % -> fr
% How the letter is closed
%\letterclosing{Hope to meet you soon,} % -> en
\letterclosing{En espérant vous rencontrer bientôt,} % -> fr
% Any enclosures with the letter
%\letterenclosure[Attached]{Curriculum Vitae}  % -> en
\letterenclosure[Ci-joint]{Curriculum Vitae} % -> fr


%-------------------------------------------------------------------------------
\begin{document}

% Print the header with above personal informations
% Give optional argument to change alignment(C: center, L: left, R: right)
\makecvheader[R]

% Print the footer with 3 arguments(<left>, <center>, <right>)
% Leave any of these blank if they are not needed
\makecvfooter
  {25 mai 2020}
  {Marc-André Labelle~~~·~~~Cover Letter}         % -> en
% {Marc-André Labelle~~~·~~~Lettre de motivation} % -> fr
  {}

% Print the title with above letter informations
\makelettertitle

%-------------------------------------------------------------------------------
%	LETTER CONTENT
%-------------------------------------------------------------------------------
\begin{cvletter}

  \lettersection{À \space propos de moi}
    Dans le cadre de mon parcours professionnel qui débute
    en ingénierie alimentaire, j'ai toujours été
    fasciné par les processus de fermentation.
    J’ai eu la chance de faire des stages dans des compagnies
    telles que la Dominion \& Grimm, Sucre Lantic et Happy Yak.
    J’y ai approfondi mes connaissances de gestion des opérations dans
    les entreprises manufacturières alimentaires.

    En tant que nouveau diplômé en génie alimentaire et futur diplômé
    en informatique, ma formation académique m’a permis de connaître
    de productions alimentaires.
    Aussi, j'ai eu le souci d'apprendre l'informatique pour exploiter
    le potentiel d'automatisation dans ces industries qui me fascinent.

  \lettersection{Pourquoid Sleeman-Unibroue?}
    Depuis que j'ai commencé mes études en génie alimentaire, j'ai toujours
    eu envie de participer à l'industrie brassicole.
    Je brasse déjà de la bière avec mon frère depuis 3 ans et je suis
    un grand amateur de bières de microbrasserie.
    Je connais bien les produits Sleeman, Unibroue et leur histoire.

  \lettersection{Pourquoi suis-je le bon candidat?}
    D'abord, j'ai un atout qu'aucun étudiant de génie chimique, alimentaire ou
    de sciences \& technologie des aliments. \textbf{Je sais programmer}.
    De plus, j'ai de l'expérience en sanitation en usine alimentaire.
    Maîtriser les équipements, leur entretien et les systèmes de nettoyage
    se fera rapidement.

\end{cvletter}


%-------------------------------------------------------------------------------
% Print the signature and enclosures with above letter informations
\makeletterclosing

\end{document}
